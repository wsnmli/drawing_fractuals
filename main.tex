%   pdflatex -shell-escape main.tex 

\documentclass[12pt]{article}

\usepackage{minted}
\usepackage{amsfonts}

\begin{document}

\begin{titlepage}
    
    {\centering
        \title{{How to Draw Fractals with C}\\}
        \author{Matthew James Lindsay}
        %\date{}
    }
        
    \clearpage\maketitle
    \thispagestyle{empty}
\end{titlepage}

\tableofcontents
\clearpage

\setcounter{page}{1}
    This set of notes assume no maths higher than GCSE level
     and absolutely no programming experience. 

\section{Elementary Math}
\subsection{Sets}
    sets can be defined by listing its elements separated by commas in curly 
    braces ie $S = \{1,2,3,4\}$

    Two sets A and B are equal iff every element of A is in B, and every element
    of B is in A. ie if $A = \{1,2,3,4\}$ and $B = \{1,2,3,4,4\}$ then $A=B$
    (repeats aren't counted in a set)
    
\subsection{Some useful sets}
    $\mathbb{N}$ the set of natural numbers $ \mathbb{N}=\{0,1,2,3,\ldots\} $
    
    $\mathbb{Z}$ the set of of all integers $\mathbb{Z} = \{\ldots,-2-1,0,1,2,3,\ldots\}$

    $\mathbb{Q}$ the set of rational numbers $\mathbb{Q} = \{\frac{a}{b} : a,b \in\mathbb{Z}, b \neq 0\}$

    $\mathbb{R}$ the set of real numbers

    $\mathbb{C}$ the set of complex numbers $\mathbb{C} = \{a+bi : a, b \in \mathbb{R}\}$
\subsection{Functions}
    functions map every element from one set of elements(domain) to an element of another set(codomain).
    for example the function $f(x) = x^2$ maps the number 2 to 4 and -3 to 9. here the domain is 
    $\mathbb{R}$ and the codmain is $\mathbb{R^+}$
\subsection{Induction}

\section{C and the Shell}
    In this course we will be using the C programing language because C is the bestest!
\subsection{Hello World}
    Type the code in the following box into your favorite text editor
\inputminted{c}{hello.c}
    save as hello.c

    at the command line type

    \texttt{gcc hello.c \&\& ./a.out}

    this is really two commands, the first \texttt{gcc hello.c} compiles our source code into
    machine language. this outputs the file \texttt{a.out} in the same directory as the 
    source code.
\subsection{defining functions}
    In C functions are defined by writing the return type, the function name, and have 
    the variables separated by commas in parentheses.

    MAJOR DIFFERENCE between C functions and maths functions, C functions have SIDE EFFECTS!!!.

\section{Drawing Geometry}
\subsection{Cartesian Geometry}
    2D Cartesian geometry is used to encode and manipulate points in 2D space
    to do this we choose some point as the origin and mark in O, choose a unit 
    vector in some direction 

    Now in computer graphics the origin is the top left pixel of the monitor 
    and 
\subsection{Drawing dots}
\subsection{for loops to draw lines}

\section{Fractal Geometry}
\subsection{Cantors Comb}
\subsection{Circles in circles}

\end{document}